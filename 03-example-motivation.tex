%---------------------------------------------------------
%	PRESENTATION BODY SLIDES
%---------------------------------------------------------
\section{Motivation} % Note all sections and subsections are automatically placed in your table of contents

%------------------------------------------------
\begin{frame}
    \frametitle{Title}
    \begin{center}
        \textbf{To use this template, you can copy and just edit/add slides!} \newline
    \end{center}

    All of the color customization occurs in the "SELECT THEME \& COLORS" section of the code. 
    There are 3 themes at \textbf{Headline and Central Footer} prepared for you. 
    Check out color design of your school to customize the theme:\newline

    \begin{center}
        \href{https://www.smu.edu.sg/about/university-brand-identity}{https://www.smu.edu.sg/about/university-brand-identity} \newline
    \end{center}

    The remainder of these slides serve as an example to show all the features you can use: footnotes, citations, bullets, buttons, sections, etc.

    \begin{center}
        {\Huge\calligra Enjoy!}
    \end{center}
\end{frame}

%------------------------------------------------
\begin{frame}
    \frametitle{Intra-frame Footnotes and Citations - Case I}
    
    Citations in beamer are slightly different with conventional cites as beamer rewrites its footnote and citation functions. 
    A common issue is the duplication of footnotes in a frame when using \textbf{footcite}. \newline

    This paper \footcite{Harshman1970}, that paper \footcite{Hitchcock1927}, and another paper \footcite{Carroll1970}. \newline

    And this paper \footcite{Harshman1970}, that paper \footcite{Hitchcock1927}, and another paper \footcite{Carroll1970} again. 
\end{frame}

%------------------------------------------------
\begin{frame}
    \frametitle{Inter-frame Footnotes and Citations - Case I}

    Another issue with \textbf{footcite} is the irritating continuation of footnote index. \newline

    This paper \footcite{Harshman1970}, that paper \footcite{Hitchcock1927}, and another paper \footcite{Carroll1970}. \newline

    And this paper \footcite{Harshman1970}, that paper \footcite{Hitchcock1927}, and another paper \footcite{Carroll1970} again. \newline

    This template provides a workaround for these issues.
\end{frame}

%------------------------------------------------
\begin{frame}
    \frametitle{Intra-frame Footnotes and Citations - Case II}
    
    Let's use customized function \textbf{firstcite} when citing a reference in a frame for the first time, and use \textbf{secondcite} for the following citations. \newline

    This paper \firstcite{Harshman1970}, that paper \firstcite{Hitchcock1927}, and another paper \firstcite{Carroll1970}. \newline

    And this paper \secondcite{Harshman1970}, that paper \secondcite{Hitchcock1927}, and another paper \secondcite{Carroll1970} again. 
\end{frame}

%------------------------------------------------
\begin{frame}
    \frametitle{Inter-frame Footnotes and Citations - Case II}

    This workaround also works like a charm for the inter-frame cases. \newline

    This paper \firstcite{Harshman1970}, that paper \firstcite{Hitchcock1927}, and another paper \firstcite{Carroll1970}. \newline

    And this paper \secondcite{Harshman1970}, that paper \secondcite{Hitchcock1927}, and another paper \secondcite{Carroll1970} again. 
\end{frame}

%------------------------------------------------
\begin{frame}
    \frametitle{Another Title}
    \framesubtitle{and a subtitle!}

    Look at the code of this slide to see how columns made this formatting look nice.

    \begin{columns}[t] % The "c" option specifies centered vertical alignment while the "t" option is used for top vertical alignment
        \begin{column}{0.5\textwidth} % Right column width
            % To add an image %
            \begin{figure}[h!]
                \centering
                %\caption{Hawkins et al, 2015}
                \includegraphics[angle=0, width=4.5cm]{hokie.png}
                %\label{Figure 1}
            \end{figure}
        \end{column}
        \begin{column}{0.5\textwidth} % Left column width
            \begin{figure}[h!]
                \centering
                %\caption{Hawkins et al, 2015}
                \includegraphics[angle=0, width=4.5cm]{hokie.png}
                %\label{Figure 1}
            \end{figure}
        \end{column}
    \end{columns}
\end{frame}

%------------------------------------------------
\begin{frame}
    \frametitle{Yet another title}
    You can use bullets too: \newline
    \begin{itemize}
        \item Like this one \newline
        \item \& this one
    \end{itemize}
\end{frame}

%------------------------------------------------
\begin{frame}
    \label{Test} %For the link button for the Appendix slide
    \frametitle{A title}

    \begin{itemize}
        \item You can also nest sub-bullets
              \begin{itemize}
                  \item Sub-bullet 1
                  \item Sub-bullet 2
                  \item Sub-bullet 3
                  \item Sub-bullet 4 \newline
              \end{itemize}
    \end{itemize}

    \textbf{Below is a button that links to a slide in the appendix}

    \begin{center}
        \hyperlink{Figure}{\beamergotobutton{Go to graphs}}
    \end{center}
\end{frame}